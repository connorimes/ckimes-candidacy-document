\tikzset{%
  app/.style    = {draw, thin, rectangle, minimum height = 2em,
    minimum width = 2em, fill=black!25},
  block/.style    = {draw, thick, rectangle, minimum height = 2.5em,
    minimum width = 2.5em},
  blockres/.style    = {draw, thick, rectangle, minimum height = 2.5em,
    minimum width = 2.5em, fill=green!25},
  biblock/.style  = {draw, thick, rectangle, minimum height = 5.5em,
    minimum width = 6em, fill=red!25},
  sum/.style      = {draw, circle, node distance = 2cm}, % Adder
  input/.style    = {coordinate}, % Input
  output/.style   = {coordinate} % Output
}

\begin{tikzpicture}[scale=1.0,transform shape, auto, thick, node distance=1.5cm, >=triangle 45]

\draw
  % Drawing the top blocks
  % node [input, name=goalaccuracy] {} 
  % node [left of=goalaccuracy, node distance=0.35mm]{}
  % node [sum, right of=goalaccuracy] (sumaccuracy) {} % negative feedback
  node [block, align=center] (featureselection) 
    {Feature\\Selection}
  node [block, right of=featureselection, align=center, node distance=3.5cm] (classifier) 
    {Classifier}
  node [blockres, above of=classifier, align=center, node distance=1.7cm] (trainingdata) 
    {Training\\Data}
;
  % Connectng lines
% \draw[->](goalaccuracy) -- node[align=center] {Timing\\Goal}(sumaccuracy);
% \draw[->](sumaccuracy) -- node[align=center] {Timing\\Error}(featureselection);
\draw[->](featureselection) -- node[align=center] {Processed\\Data}(classifier);
\draw[->](trainingdata) -- (classifier);

% Draw software system
\draw
  node [biblock, right of=classifier, node distance=4.2cm, align=center] (system)
    {\\System\\\\\\}
;
\draw
  node [app, right of=classifier, node distance=4.2cm, align=center, yshift=-0.5cm] (software)
    {Application}
;

% lines from translators to software
\draw[->](classifier.east) -- node [name=ka,align=center]{DVFS\\Frequency} (classifier.east -| system.west);

% Connectng lines
\coordinate (feedbackup) at ([yshift=-0.5cm]featureselection.south);
\draw (system.west |- feedbackup) -| node [near end,align=center] {PCM\\Sample} (feedbackup);
\draw[->](feedbackup) -- node[pos=0.99] {} (featureselection);

\end{tikzpicture}