\documentclass{article}
\usepackage{subfigure,epsfig,amsfonts}
\usepackage{amsthm}
\usepackage{natbib}
\bibliographystyle{plain}

\usepackage[normalem]{ulem}
\usepackage{float}
\usepackage{url}
\usepackage{makeidx}
\usepackage{graphicx}
\usepackage{graphics}
\usepackage{multicol}
\usepackage{xspace}
% \usepackage{chngpage}
%\usepackage{narrow}
\usepackage{fullpage}
%\usepackage{subfigure}
\usepackage{subfig}
\usepackage{multirow}
\usepackage{rotating}
\usepackage{color}
%\usepackage{algorithmic2e}
\usepackage{algpseudocode}
\usepackage{algorithm}
\usepackage{savesym}
\usepackage{amsmath}
\usepackage{amssymb,latexsym}
\everymath{\displaystyle}
\savesymbol{iint} 
\usepackage{txfonts} 
\restoresymbol{TXF}{iint}
%\usepackage{mathptmx} % amsmath makes the paper longer
\usepackage{mathrsfs}
\usepackage{leading}
%\usepackage{algorithmic}
%\usepackage{algorithm}
%\usepackage{fullpage}
%\usepackage[font={small,bf},nooneline]{caption}
\usepackage{hyphenat}
\usepackage{leading}
\let\labelindent\relax
\usepackage{enumitem}
%\usepackage[natbib=true,backend=bibtex,firstinits=true,style=numeric-comp,sorting=nyt,defernumbers,maxnames=10,maxcitenames=2,doi=false,isbn=false,url=false]{biblatex}
\usepackage{balance}
\usepackage{pbox}

\definecolor{mygreen}{rgb}{0,0.6,0}
\definecolor{mygray}{rgb}{0.5,0.5,0.5}
\definecolor{mymauve}{rgb}{0.58,0,0.82}
\usepackage{listings}
\lstset{ %
  backgroundcolor=\color{white},   % choose the background color; you must add \usepackage{color} or \usepackage{xcolor}
  basicstyle=\scriptsize\ttfamily, % the size of the fonts that are used for the code
  breakatwhitespace=false,         % sets if automatic breaks should only happen at whitespace
  breaklines=true,                 % sets automatic line breaking
  captionpos=b,                    % sets the caption-position to bottom
  commentstyle=\color{mygreen},    % comment style
  deletekeywords={...},            % if you want to delete keywords from the given language
  escapeinside={\%*}{*)},          % if you want to add LaTeX within your code
  extendedchars=true,              % lets you use non-ASCII characters; for 8-bits encodings only, does not work with UTF-8
  frame=leftline,                  % adds a frame around the code
  keepspaces=true,                 % keeps spaces in text, useful for keeping indentation of code (possibly needs columns=flexible)
  keywordstyle=\color{blue},       % keyword style
  morekeywords={*,...},            % if you want to add more keywords to the set
  numbers=left,                    % where to put the line-numbers; possible values are (none, left, right)
  numbersep=5pt,                   % how far the line-numbers are from the code
  numberstyle=\tiny\color{mygray}, % the style that is used for the line-numbers
  rulecolor=\color{black},         % if not set, the frame-color may be changed on line-breaks within not-black text (e.g. comments (green here))
  showspaces=false,                % show spaces everywhere adding particular underscores; it overrides 'showstringspaces'
  showstringspaces=false,          % underline spaces within strings only
  showtabs=false,                  % show tabs within strings adding particular underscores
  stepnumber=1,                    % the step between two line-numbers. If it's 1, each line will be numbered
  stringstyle=\color{black},     % string literal style
  tabsize=1,                       % sets default tabsize to 2 spaces
  title=\lstname                   % show the filename of files included with \lstinputlisting; also try caption instead of title
}

\usepackage{pgfplots}
% options for pgfplots
\pgfplotsset{compat=1.8,compat/show suggested version=false}
\usetikzlibrary{calc,trees,arrows,patterns,plotmarks,shapes,snakes,er,3d,automata,backgrounds,topaths,decorations.pathmorphing,decorations.markings}
%\pgfplotsset{compat=newest}
\pgfplotsset{
   /pgfplots/bar  cycle  list/.style={/pgfplots/cycle  list={%
        {black,fill=black!30!white,mark=none},%
        {black,fill=red!30!white,mark=none},%
        {black,fill=green!30!white,mark=none},%
        {black,fill=yellow!30!white,mark=none},%
        {black,fill=brown!30!white,mark=none},%
     }
   },
}
% begin of externalization
\usetikzlibrary{external}
\tikzexternalize[prefix=out/]
\tikzexternalize
% don't externalize todonotes
%\makeatletter
%\renewcommand{\todo}[2][]{\tikzexternaldisable\@todo[#1]{#2}\tikzexternalenable}
%\makeatother
% end of externalization
\usetikzlibrary{patterns}
\usepgfplotslibrary{groupplots}
\pgfplotsset{
every axis label/.append style={font=\footnotesize},
tick label style={font=\footnotesize},
}
%\usepackage[normalem]{ulem}
%\setlength{\abovecaptionskip}{2pt plus 2pt minus 2pt}
\setlist{noitemsep,topsep=0pt}

%% Use these commands to set biographic information for the title page:
\title{Balancing Performance and Energy on Server-class Systems}
\author{Connor Kelly Imes}


% some useful shortcuts
\newcommand{\ie}{\textit{i.e., }}
\newcommand{\eg}{\textit{e.g., }}
\newcommand{\etal}{et al.\ }
\newcommand{\CC}{C\nolinebreak\hspace{-.05em}\raisebox{.5ex}{\tiny\bf +}\nolinebreak\hspace{-.10em}\raisebox{.5ex}{\tiny\bf +}}

% units for results
\newcommand{\us}{\,$\mu$s}
\newcommand{\ms}{\,ms}
\newcommand{\KB}{\,KB}
\newcommand{\MB}{\,MB}
\newcommand{\GB}{\,GB}
\newcommand{\MHz}{\,MHz}
\newcommand{\GHz}{\,GHz}

% \newcommand{\SYSTEM}{POET}
% \newcommand{\system}{poet}
% new latex commands:
%   Remove long section
\newcommand{\PUNT}[1]{}
%   Label work to be done
\definecolor{gray}{gray}{0.75}
\newcommand{\TODO}[1]{\textcolor{gray}{\textbf{\ [TODO:\ #1]\ }}}
\newcommand{\TR}[1]{#1}
%\newcommand{\TR}[1]{}
%\newcommand{\TODO}[1]{}
\newcommand{\FIX}[1] {\textcolor{red}{\textbf{\ [FIX:\ #1]\ }}}
%   Referencing various pieces of the document:
\newcommand{\figref}[1]{Figure~\ref{fig:#1}}
\newcommand{\figsref}[2]{Figures~\ref{fig:#1} and~\ref{fig:#2}}
\newcommand{\figrref}[2]{Figures~\ref{fig:#1}--\ref{fig:#2}}
\newcommand{\secref}[1]{Chapter~\ref{sec:#1}}
\newcommand{\secsref}[2]{Sections~\ref{sec:#1} and~\ref{sec:#2}}
\newcommand{\eqnref}[1]{Eqn.~\ref{eqn:#1}}
\newcommand{\eqnsref}[2]{Eqns.~\ref{eqn:#1} and~\ref{eqn:#2}}
\newcommand{\eqnrref}[2]{Eqns.~\ref{eqn:#1}--\ref{eqn:#2}}
\newcommand{\insref}[1]{Instruction~\ref{ins:#1}}
\newcommand{\tblref}[1]{Table~\ref{tbl:#1}}
\newcommand{\appref}[1]{Appendix~\ref{app:#1}}
\newcommand{\algoref}[1]{Algorithm~\ref{algo:#1}}

% Custom hyphenation rules
\hyphenation{Ang-strom}

%\DeclareMathOperator{\minimize}{minimize}
%\DeclareMathOperator{\st}{s.t.}
%\DeclareMathOperator*{\argmin}{argmin}
%\DeclareMathOperator*{\argmax}{argmax}
\newcommand{\argmin}{\arg\!\min}
\newcommand{\argmax}{\arg\!\max}
\newcommand{\minimize}{minimize}
\newcommand{\optimize}{optimize}
\newcommand{\st}{s.t.}

\newcommand{\bench}[1]{\mbox{\texttt{#1}}}
\newcommand{\function}[1]{\mbox{\texttt{#1}}}
\newcommand{\struct}[1]{\textit{#1}}
\newcommand{\utility}[1]{\mbox{\texttt{#1}}}


%\usepackage[firstpage]{draftwatermark}

%-------------------------------------------------------------------------
\begin{document}

\date{}
\maketitle
%\omittitle

\tableofcontents
% \listoffigures
% \listoftables

\section{Thesis Statement}

This thesis explores two problems in balancing software performance and energy consumption in server-class systems: 1) meeting application performance goals while minimizing energy consumption, and 2) running applications as energy-efficiently as possible (maximizing the work to energy consumption ratio).
We address the first problem by using control theory to meet performance goals while tuning system settings, such as DVFS frequencies, core allocations, or hardware power caps, to optimize energy consumption.
We treat the second as a classification problem, using low-level hardware counter metrics with machine learning approaches to predict the most energy-efficient system settings at runtime.

\section{Introduction}

In computing systems, performance and energy consumption are conflicting goals -- increasing performance requires a super-linear increase in power consumption.
Thermal design constraints limit both the maximum and average performance that a computing system can achieve.
Fortunately, modern systems expose knobs that allow hardware and software to trade performance and power/energy consumption, so that a desired balance can be reached.
For example, most systems provide a mechanism to control dynamic voltage and frequency scaling (DVFS), trading processor clock rate and power consumption for changes in throughput.
Another common approach is to change the allocation of the number of compute cores on multicore systems, allowing unallocated cores to enter low-power sleep states, or even shut down altogether to save power/energy.

We can characterize the performance and power consumption of applications in different system configurations by testing them in each configuration and recording the behavior.
Performance and power form a non-linear, convex tradeoff space, where each system configuration, \eg DVFS setting, produce a unique application performance and system power values.
\TODO{Show an example.}

% In some cases, the goal is simply to run software as fast as possible, and as long as energy consumption is not a concern, this is acceptable.
As power and energy become first-class concerns in computing systems ranging from low-power embedded to large scale high-performance computing (HPC), there is an increasing need for software to manage hardware resource allocations to achieve a desirable balance in performance and power/energy consumption.
Heuristic approaches rely on assumptions about the performance/power tradeoff space of an application and system that we have found are not portable between applications and systems (as we demonstrated in embedded systems) \cite{Imes2014}.
For example, the classic race-to-idle heuristic can achieve low energy consumption on one system, but poor on another compared with an approach that never idles (or even a pace-to-idle approach).
An interesting observation by Kim \etal is, ``Unless the function of a given machine is a straight line originating from the idle state, (0, $P_{idle}$), an idling heuristic is never optimal for all instances'' \cite{kim-cpsna2015}.
The more convex the tradeoff space, the further from optimal any idling approach is.

The lack of both portability and optimality of heuristics demand that we develop more general approaches to addressing the problems in balancing performance and power.
Hoffmann proposes a ``self-aware'' computing model that uses a closed-loop feedback design to ``observe, decide, and act'' \TODO{citation}.
\TODO{Show a generic closed-loop feedback control (SEEC-like) system.}
This model is more portable than heuristics because it includes an observation step that measures behavior at runtime rather than relying strictly on assumptions made offline.
We build on this high-level model and use feedback systems that measure application and system behavior during runtime to make changes to resource allocations as new information becomes available.
Furthermore, the general design of our feedback systems, although still relying on some offline characterization of application and system behavior, are not dependent on extremely accurate or complete models, instead relying on runtime measurements to adapt.


This thesis addresses two common approaches in managing performance and power consumption.
The first approach is meeting an application performance goal while attempting to minimize the total energy consumption.
We formulate this as a constrained optimization problem, using control theory to meet the performance goal, and linear optimization to select the optimal pair of system configurations to use that satisfy the performance constraint.
This is practical since in the aforementioned work, Kim \etal prove that an optimal solution to this constrained optimization problem requires, at most, two configurations from the performance/power tradeoff space \cite{kim-cpsna2015}.
The second approach addresses the problem of running software as energy-efficiently as possible, \ie maximizing the ration of work completed to energy consumed.
By treating this as a classification problem, we can use samples from low-level hardware counters to drive well-understood machine learning approaches that predict the most energy-efficient configuration to use, without requiring the software to provide its own instrumentation.

\section{Performance and Energy Management using Control Theory}

We first address the problem of meeting application performance goals while minimizing energy consumption.
Many applications are subject to soft performance constraints -- for example, streaming applications often consume data from other sources like sensors, or must meet QoS requirements when serving client requests.
In this project, we restrict the application domain to streaming-like applications with high-level loops, to which we apply the concept of Heartbeats to measure and record application performance and system power behavior \cite{icac2010heartbeats}.
Practically, this involves adding a small amount of instrumentation into the applications to capture these metrics at runtime.
This is the \emph{observe} component of a SEEC-like system.
To support this observation requirement, we complement Heartbeats with the EnergyMon interface for capturing power/energy data in a portable manner \cite{energymon}.\footnote{EnergyMon with Java and Rust bindings and abstractions available at https://github.com/energymon/}

Unlike heuristics, control theory provides a formal approach for meeting goals in dynamic systems.
Control-theoretic approaches can adapt despite non-trivial errors in their design parameters and guarantee convergence under certain bounds.
% There has been significant prior work \TODO{describe} in using designing software controllers, but most lack portability -- they are designed for specific systems and applications.
The \emph{decide} component of a SEEC-like system is described by the following controllers, POET and CoPPer.
Finally, the \emph{act} component is platform-specific, and typically involves actuating system settings like core allocation and processor DVFS frequencies or power limits.


\subsection{POET}

Using \cite{Imes2014} as motivation for needing a portable approach to minimizing energy under performance constraints, we designed the Performance with Optimal Energy Toolkit (POET).
POET meets soft real-time constraints and minimizes energy consumption by actuating DVFS frequencies and core allocation \cite{POET}.
While POET was originally evaluated on embedded systems, we later evaluated it on a dual socket, 32-logical-core server-class system \cite{POETMCSoC}.
Its design, shown in \figref{poet-runtime}, reflects the SEEC model.

\begin{figure}[t]
  \begin{centering}
    \tikzset{%
  app/.style    = {draw, thin, rectangle, minimum height = 2em,
    minimum width = 2em, fill=black!25},
  block/.style    = {draw, thick, rectangle, minimum height = 2.5em,
    minimum width = 2.5em},
  blockres/.style    = {draw, thick, rectangle, minimum height = 2.5em,
    minimum width = 2.5em, fill=green!25},
  biblock/.style  = {draw, thick, rectangle, minimum height = 5.5em,
    minimum width = 6em, fill=red!25},
  sum/.style      = {draw, circle, node distance = 2cm}, % Adder
  input/.style    = {coordinate}, % Input
  output/.style   = {coordinate} % Output
}

\begin{tikzpicture}[scale=1.0,transform shape, auto, thick, node distance=1.5cm, >=triangle 45]

\draw
  % Drawing the top blocks
  node [input, name=goalaccuracy] {} 
  node [left of=goalaccuracy, node distance=0.35mm]{}
  node [sum, right of=goalaccuracy] (sumaccuracy) {} % negative feedback
  node [block, right of=sumaccuracy, align=center, node distance=2.6cm] (controlaccuracy) 
    {Controller}
  node [block, right of=controlaccuracy, align=center, node distance=3.5cm] (translateaccuracy) 
    {Optimizer}
  node [blockres, above of=translateaccuracy, align=center, node distance=1.7cm] (resourcefile) 
    {Resource\\Specification}
;
  % Connectng lines
\draw[->](goalaccuracy) -- node[align=center] {Performance\\Goal}(sumaccuracy);
\draw[->](sumaccuracy) -- node[align=center] {Performance\\Error}(controlaccuracy);
\draw[->](controlaccuracy) -- node[align=center] {Generic\\Control\\Signal}(translateaccuracy);
\draw[->](resourcefile) -- (translateaccuracy);

% Draw software system
\draw
  node [biblock, right of=translateaccuracy, node distance=4.2cm, align=center] (system)
    {\\System\\\\\\}
;
\draw
  node [app, right of=translateaccuracy, node distance=4.2cm, align=center, yshift=-0.5cm] (software)
    {Application}
;

% lines from translators to software
\draw[->](translateaccuracy.east) -- node [name=ka,align=center]{Resource\\Schedule} (translateaccuracy.east -| system.west);

% Connectng lines
\coordinate (feedbackup) at ([yshift=-0.6cm]sumaccuracy.south);
\draw (software.west |- feedbackup) -| node [near end,align=center] {Performance\\Feedback} (feedbackup);
\draw[->](feedbackup) -- node[pos=0.99] {$-$} (sumaccuracy);

\end{tikzpicture}
    \caption{Overview of the POET runtime.}
    \label{fig:poet-runtime}
  \end{centering}
\end{figure}

In POET, the user provides a \emph{performance goal}, $P_{ref}$, to the controller; the user or a systems developer provides it with a \textbf{resource specification}, which is a normalized table of performance and power behavior derived from a characterization of a representative application.
At fixed work intervals called window periods, the controller computes a new \emph{resource schedule} that is applied over the following window period.

The schedule is computed by first calculating the \emph{performance error}, $e(t)$, between the performance goal, and the observed application performance (\emph{performance feedback}), $p_m(t)$:
\begin{eqnarray}
  e(t) = P_{ref} - p_m(t)
  \label{eqn:error}
\end{eqnarray}
The \textbf{controller} uses this error to compute a new speedup value (\emph{generic control signal}), $s(t)$, so that the speedup cancels the error:
\begin{equation}
  s(t) = s(t-1) + (1-p) \cdot \frac{e(t)}{b(t)} 
  \label{eqn:poet-control}
\end{equation}
Here $p$ is a configurable \emph{pole} value such that $0 \le p < 1$, which affects how responsive the controller is to system noise or changes in application behavior.
A small value of $p$ causes the controller to respond quickly, but is also more susceptible to noise.
A larger $p$ slows the response, but ensures robustness and is more useful in noisy systems.

Because the POET's resource specification is normalized to $c=0$ (described shortly), the controller must estimate the application's base speed, $b(t)$, to bridge the gap between speedup and actual performance.
Every application has its own base speed, whose value may change at runtime as the application progresses through different \emph{phases}.
POET updates its base speed estimate at every control iteration using a Kalman filter~\cite{welch2006kalman}.

Once $s(t)$ is computed, POET's \textbf{optimizer} searches the resource specification to find the pair of configurations that provides the optimal energy consumption while respecting the performance (speedup) constraint.
There are $C$ configurations in the resource specification, and by convention we number the configurations from $0$ to $C-1$.
Configuration $c = 0$ is the lowest-performance/lowest-power configuration where the least amount of resources are allocated.
Configuration $C-1$ is the highest-performance/highest-power configuration where the maximum amount of resources are allocated.
Every configuration $c$ has normalized performance (speedup) $s_c$ and normalized power consumption (powerup) $p_c$ values.

In computing the resource schedule, the optimizer assigns each configuration $c$ an execution time $\tau_c$, ensuring that the $W$ iterations complete while the total energy is minimized, where $W$ is the window period.
This optimization problem is formulated as a linear program, solved in $O(C^2)$ time:
\begin{eqnarray}
\minimize && \sum_{c=0}^{C-1} \tau_c \cdot p_c \label{eqn:poet-power} \\
\st %&& \nonumber\\
&& \sum_{c=0}^{C-1} \tau_c \cdot s_c \cdot b(t) = W \label{eqn:poet-work} \\
&& \sum_{c=0}^{C-1} \tau_c =  \tau \label{eqn:poet-deadline} \\
&& 0 \le \tau_c \le \tau, \qquad \forall c \in \{0,\ldots,C-1\} \label{eqn:poet-time}
\end{eqnarray}
The resulting schedule consists of, at most, two configurations with $\tau_c > 0$.

This control and optimization approach provides formal guarantees about steady-state convergence and robustness while remaining independent of any particular application or system.
We demonstrated POET's ability to meet performance goals and adapt to changes in application phases at runtime by executing the \bench{x264} video encoding application using an input that is a combination of three videos of varying encoding difficulty.

\figref{poet-phases-default} is a time series of job performance and system power consumption when running \bench{x264} without any control in the highest resource configuration ($C-1$).
Performance is normalized to the maximum recorded value.
Frames that are easier to encode (higher performance) take less time, indicating that it would require fewer system resources to meet a performance goal than a frame with lower performance.
Input phase changes occur at frames 1,500 and 3,000, where there are observable changes in performance.
The first and third phases have similar performance behavior, and thus are about the same level of encoding difficulty; the second phase is easiest to encode as it has higher average performance.

\begin{figure}[t]
  \begin{tikzpicture}
\begin{centering}

\definecolor{s1}{RGB}{228, 26, 28}
\definecolor{s2}{RGB}{55, 126, 184}
\definecolor{s3}{RGB}{77, 175, 74}
\definecolor{s4}{RGB}{152, 78, 163}
\definecolor{s5}{RGB}{255, 127, 0}

\begin{groupplot}[
    group style={
        group name=plots,
        group size=1 by 2,
        xlabels at=edge bottom,
        xticklabels at=edge bottom,
        vertical sep=5pt
    },
height=3.5cm,
width=0.95\columnwidth,
xmajorgrids,
ymajorgrids,
grid style={dashed},
xmin=0,
xmax=4500,
yticklabel pos=left,
enlargelimits=false,
tick align = outside,
tick style={white},
xticklabel shift={-5pt},
yticklabel shift={-5pt},
ylabel shift={-2pt},
ylabel style={align=center},
unbounded coords=jump,
]

\nextgroupplot[ylabel={\footnotesize Performance \\ (Normalized)}, % Performance
xtick={0,500,1000,1500,2000,2500,3000,3500,4000,4500},
ytick={0,0.2,0.4,0.6,0.8,1.0,1.2},
yticklabels={,0.2,0.4,0.6,0.8,1.0,1.2},
yticklabel style={font=\footnotesize},
ymin=0,
ymax=1.0,
% legend entries={{\footnotesize $\mathsf{Server}$}},
% legend style={draw=none,at={(0.5,1.35)},anchor=north,legend columns=4,line width=5pt},
]
\addplot[thick, solid, color=s2] table[x index=0,y index=1,col sep=tab] {img/clover/x264-phases-default-clover.txt};
\addplot[thick, dashed, black] coordinates {(1500,0) (1500, 2)};
\addplot[thick, dashed, black] coordinates {(3000,0) (3000, 2)};


\nextgroupplot[ylabel={\footnotesize Power \\ (Watts)}, % Power
ytick={0,50,100,150,200,250},
yticklabels={,50,100,150,200,250},
yticklabel style={font=\footnotesize},
ymin=50,
ymax=250,
xlabel={\footnotesize $time$ [frame]},
xlabel near ticks,
xtick={0,500,1000,1500,2000,2500,3000,3500,4000,4500},
xticklabels={0,,,1500,,,3000,,,4500},
xticklabel style={font=\footnotesize},
]
\addplot[thick, solid, color=s1] table[x index=0,y index=2,col sep=tab] {img/clover/x264-phases-default-clover.txt};
\addplot[thick, dashed, black] coordinates {(1500,0) (1500, 250)};
\addplot[thick, dashed, black] coordinates {(3000,0) (3000, 250)};

\end{groupplot}
\end{centering}

\end{tikzpicture}
  \caption{Processing x264 input with distinct phases in an uncontrolled setting.}
  \label{fig:poet-phases-default}
\end{figure}

\begin{figure}[t]
  \begin{tikzpicture}
\begin{centering}

\definecolor{s1}{RGB}{228, 26, 28}
\definecolor{s2}{RGB}{55, 126, 184}
\definecolor{s3}{RGB}{77, 175, 74}
\definecolor{s4}{RGB}{152, 78, 163}
\definecolor{s5}{RGB}{255, 127, 0}

\begin{groupplot}[
    group style={
        group name=plots,
        group size=1 by 2,
        xlabels at=edge bottom,
        xticklabels at=edge bottom,
        vertical sep=5pt
    },
height=3.5cm,
width=0.95\columnwidth,
xmajorgrids,
ymajorgrids,
grid style={dashed},
xmin=0,
xmax=4500,
yticklabel pos=left,
enlargelimits=false,
tick align = outside,
tick style={white},
xticklabel shift={-5pt},
yticklabel shift={-5pt},
ylabel shift={-2pt},
ylabel style={align=center},
unbounded coords=jump,
]

\nextgroupplot[ylabel={\footnotesize Latency \\ (Normalized)}, % Performance
xtick={0,500,1000,1500,2000,2500,3000,3500,4000,4500},
ytick={0.0,0.5,1.0,1.5,2.0},
yticklabels={,0.5,1.0,1.5,2.0},
yticklabel style={font=\footnotesize},
ymin=0,
ymax=2,
% legend entries={{\footnotesize $\mathsf{Server}$}},
% legend style={draw=none,at={(0.5,1.4)},anchor=north,legend columns=4,line width=5pt},
]
\addplot[thick, solid, color=s2] table[x index=0,y index=1,col sep=tab] {img/POET/x264-phases-clover-seec.txt};
\addplot[thick, solid, black] coordinates {(0, 1) (4500, 1)};
\addplot[thick, dashed, black] coordinates {(1500,0) (1500, 2)};
\addplot[thick, dashed, black] coordinates {(3000,0) (3000, 2)};


\nextgroupplot[ylabel={\footnotesize Power \\ (Watts)}, % Power
ytick={0,50,100,150,200,250},
yticklabels={,50,100,150,200,250},
yticklabel style={font=\footnotesize},
ymin=50,
ymax=250,
xlabel={\footnotesize $time$ [frame]},
xlabel near ticks,
xtick={0,500,1000,1500,2000,2500,3000,3500,4000,4500},
xticklabels={0,,,1500,,,3000,,,4500},
xticklabel style={font=\footnotesize},
]
\addplot[thick, solid, color=s1] table[x index=0,y index=2,col sep=tab] {img/POET/x264-phases-clover-seec.txt};
\addplot[thick, dashed, black] coordinates {(1500,0) (1500, 250)};
\addplot[thick, dashed, black] coordinates {(3000,0) (3000, 250)};

\end{groupplot}
\end{centering}

\end{tikzpicture}

  \caption{Processing x264 input with distinct phases using POET.}
  \label{fig:poet-phases-x264}
\end{figure}

In \figref{poet-phases-x264}, we run with POET enabled, using a performance target that is about half of the system's maximum capacity.
POET takes initial performance measurements during the first window period of 100 frames, then begins changing system resources to meet the performance goal.
After the end of the third window period, around frame 300, the soft performance goal is respected for the remainder of the runtime.
Performance is still somewhat noisy due to the unpredictable behavior of \bench{x264}, as can be observed in the uncontrolled execution, with slight dips and peaks appearing at the beginning of phase changes before POET has time to adapt.

POET has also been used as the foundation for other controllers and projects.
We expanded on POET to create Bard, which adds the ability to instead meet soft power constraints and maximize performance \cite{Bard}.\footnote{Both POET and Bard are available at http://poet.cs.uchicago.edu/}
Farrell and Hoffmann build on POET by additionally changing application accuracy to provide hard real-time guarantees \cite{meantime}.
In work still under review, Mishra \etal combine POET with machine learning to reduce the amount of offline work needed to generate resource specifications, and update both the specification and pole value online as new application behavior is learned.
POET was also the starting point in an ongoing project called Proteus, in which we are developing a programming language called FAST that allows programmers to specify more general constraints and optimizations called \emph{intents} -- both system and application knobs can be adjusted to satisfy the intents.


\subsection{CoPPer}

As hardware manufacturers move away from software-specified DVFS settings, we instead propose using software-defined, hardware-enforced power capping as the actuation mechanism for controllers to balance performance and power/energy consumption.
We therefore proposed CoPPer, \textbf{Co}ntrol \textbf{P}erformance with \textbf{P}ow\textbf{er}.\footnote{CoPPer and related tools will be available at https://github.com/powercap/}
In CoPPer, we address the same problem of minimizing energy consumption under soft performance constraints, with the hardware power capping implementation performing DVFS optimization ``under the hood'' for us, avoiding the need for the software controller to compute an optimal schedule.
For this project, we do not manage core allocation.
We evaluated CoPPer on the same server-class system as POET, using Intel Running Average Power Limit (RAPL) as the power capping implementation \cite{RAPL}.
Specifically, we use the short term Package-level power constraints on each of the two system processor sockets.

\begin{figure}[t]
  \begin{centering}
    \tikzset{%
  app/.style    = {draw, thin, rectangle, minimum height = 2em,
    minimum width = 2em, fill=black!25},
  block/.style    = {draw, thick, rectangle, minimum height = 2.5em,
    minimum width = 2.5em},
  blockres/.style    = {draw, thick, rectangle, minimum height = 2.5em,
    minimum width = 2.5em, fill=green!25},
  biblock/.style  = {draw, thick, rectangle, minimum height = 5.5em,
    minimum width = 6em, fill=red!25},
  sum/.style      = {draw, circle, node distance = 2cm}, % Adder
  input/.style    = {coordinate}, % Input
  output/.style   = {coordinate} % Output
}

\begin{tikzpicture}[scale=1.0,transform shape, auto, thick, node distance=1.5cm, >=triangle 45]

\draw
  % Drawing the top blocks
  node [input, name=goalaccuracy] {} 
  node [left of=goalaccuracy, node distance=0.35mm]{}
  node [sum, right of=goalaccuracy] (sumaccuracy) {} % negative feedback
  node [block, right of=sumaccuracy, align=center, node distance=2.9cm] (controlaccuracy) 
    {~Controller~}
  % node [block, right of=controlaccuracy, align=center, node distance=3.0cm] (translateaccuracy) 
  %   {Mapper}
  node [blockres, above of=controlaccuracy, align=center, node distance=1.8cm] (resourcefile) 
    {Min / Max\\Power}
;
  % Connectng lines
\draw[->](goalaccuracy) -- node[align=center] {Performance\\Goal}(sumaccuracy);
\draw[->](sumaccuracy) -- node[align=center] {Performance\\Error}(controlaccuracy);
% \draw[->](controlaccuracy) -- node[align=center] {Generic\\Control\\Signal}(translateaccuracy);
\draw[->](resourcefile) -- (controlaccuracy);

% Draw software system
\draw
  node [biblock, right of=controlaccuracy, node distance=4.2cm, align=center] (system)
    {\\System\\\\\\}
;
\draw
  node [app, right of=controlaccuracy, node distance=4.2cm, align=center, yshift=-0.5cm] (software)
    {Application}
;

% lines from translators to software
\draw[->](controlaccuracy.east) -- node [name=ka,align=center]{Power Cap} (controlaccuracy.east -| system.west);

% Connectng lines
\coordinate (feedbackup) at ([yshift=-0.6cm]sumaccuracy.south);
\draw (software.west |- feedbackup) -| node [near end,align=center] {\\Performance\\Feedback} (feedbackup);
\draw[->](feedbackup) -- node[pos=0.99] {$-$} (sumaccuracy);

\end{tikzpicture}
    \caption{Overview of the CoPPer runtime.}
    \label{fig:copper-runtime}
  \end{centering}
\end{figure}

\figref{copper-runtime} demonstrates CoPPer's design, which also reflects the SEEC model.
Note the absence of an \textbf{optimizer} and \textbf{resource specification}.
Unlike POET, CoPPer does not require a user or developer-specified model based on an application characterization -- only minimum and maximum power caps are needed.
While CoPPer uses a similar control-theoretic approach as POET, it introduces a \emph{gain limit} term into the speedup formulation.
In layman's terms, the gain limit ``pulls'' down the power cap to prevent over-allocating power when it is not beneficial.

As in POET, the CoPPer first calculates the error, $e(t)$, between desired and actual performance and estimates the base speed, $b(t)$ using a Kalman filter.
The speedup formulation used in CoPPer is then:
\begin{eqnarray}
  s(t) = max\left(1, gain(t) \cdot min\left(s(t-1) + \frac{e(t)}{b(t)}, \frac{U_{max}}{U_{min}}\right)\right)
  \label{eqn:copper-speedup-control}
\end{eqnarray}
where $U_{min}$ and $U_{max}$ are the system's minimum and maximum allowable power caps, respectively, and $0 < gain(t) <= 1$.
Using $U_{min}$ and $U_{max}$ to clamp the upper bound of $s(t)$ before applying the gain prevents slow controller response when the performance goal is not achievable (in control theory terminology, this is called an anti-windup mechanism).
Note that the speedup signal is now strictly internal to the \textbf{controller}, and is translated to a \emph{power cap} by scaling $s(t)$ by $U_{min}$.
To maintain convergence guarantees, $gain(t)$ is set to $1$ until the controller settles, regardless of the user configuration.

To formally describe the gain, we quote from the CoPPer paper (still under review):
\begin{quote}
Intuitively, if the performance error value [...] is low, then the system
has converged to the performance target and the speedup signal should
remain where it is.  However, if error values are high but the
\emph{difference} in error values between iterations is low, the controller
has settled, but the performance target is not achievable.  It
therefore may be beneficial to reduce the speedup, and thus the power.
Speedup is reduced by setting gain to:
\begin{eqnarray}
  gain(t) = 1 - \alpha_c \cdot e_{ns}(t) \cdot \Delta e_{ns}(t)
  \label{eqn:cost-pole}
\end{eqnarray}
where $\alpha_c$ ($0 \le \alpha_c < 1$) is the \emph{gain limit}, a
constant that controls how low the gain can go, and:
\begin{eqnarray}
  e_n(t) = \frac{|e(t)|}{P_{ref}}
  \label{eqn:en} \\
  \Delta e_n(t) = \left| e_n(t-1) - e_n(t) \right|
  \label{eqn:den} \\
  e_{ns}(t) = 1 - \frac{1}{e_{n}(t)+1}
  \label{eqn:ens} \\
  \Delta e_{ns}(t) = \frac{1}{\Delta e_{n}(t)+1}
  \label{eqn:dens}
\end{eqnarray}
Since $P_{ref}$ is the performance target, $e_n(t)$ is the absolute normalized performance error.
$\Delta e_n(t)$ in \eqnref{den} is the absolute change in $e_n(t)$ since the previous
iteration.  \eqnsref{ens}{dens} compute values $e_{ns}(t)$ and $\Delta
e_{ns}(t)$, which determine how much impact $e_n(t)$ and $\Delta
e_{n}(t)$ have on reducing the speedup.  Both $e_{ns}(t)$ and $\Delta
e_{ns}(t)$ lay in the unit circle.  As normalized performance error
$e_n(t)$ approaches 0, $e_{ns}(t)$ also approaches 0, which reduces
the impact of the gain limit in \eqnref{cost-pole}.  Conversely, if
the error is high, $e_{ns}(t)$ approaches 1 and the gain limit will
have a greater impact on the change speedup.  As the change in
normalized performance error $\Delta e_{n}(t)$ approaches 0, $\Delta
e_{ns}(t)$ approaches 1, thus increasing the gain limit's impact in
\eqnref{cost-pole}.  Conversely, if the change in error is high,
$\Delta e_{ns}(t)$ approaches 0 and the gain limit will have less
impact on the change in speedup.  Therefore, \eqnref{cost-pole}
reduces the speedup signal by a factor of at most $\alpha_c$, with the
greatest change in speedup occurring when the absolute performance
error $e_n(t)$ is high and the absolute change in error $e_n(t)$ is
low.  Setting $\alpha_c=0$ disables the gain limit entirely,
corresponding to $gain(t) =1 $ in \eqnref{copper-speedup-control}.
\end{quote}

\begin{figure}[t]
  \begin{tikzpicture}
\begin{centering}

\definecolor{s1}{RGB}{228, 26, 28}
\definecolor{s2}{RGB}{55, 126, 184}
\definecolor{s3}{RGB}{77, 175, 74}
\definecolor{s4}{RGB}{152, 78, 163}
\definecolor{s5}{RGB}{255, 127, 0}

\begin{groupplot}[
    group style={
        group name=plots,
        group size=1 by 2,
        xlabels at=edge bottom,
        xticklabels at=edge bottom,
        vertical sep=5pt
    },
height=3.5cm,
width=0.95\columnwidth,
xmajorgrids,
ymajorgrids,
grid style={dashed},
xmin=0,
xmax=4500,
yticklabel pos=left,
enlargelimits=false,
tick align = outside,
tick style={white},
xticklabel shift={-5pt},
yticklabel shift={-5pt},
ylabel shift={-2pt},
ylabel style={align=center},
unbounded coords=jump,
]

\nextgroupplot[ylabel={\footnotesize Performance \\ (Normalized)}, % Performance
xtick={0,500,1000,1500,2000,2500,3000,3500,4000,4500},
ytick={0.0,0.5,1.0,1.5,2.0},
yticklabels={,0.5,1.0,1.5,2.0},
yticklabel style={font=\footnotesize},
ymin=0,
ymax=2,
% legend entries={{$\mathsf{DVFS}$},{$\mathsf{\SYSTEM{}}$}},
% legend style={draw=none,at={(0.5,1.4)},anchor=north,legend columns=4,line width=5pt},
]
\addplot[thick, solid, color=s2] table[x index=0,y index=1,col sep=tab] {img/CoPPer/x264-phases-clover-copper.txt};
\addplot[thick, solid, black] coordinates {(0, 1) (4500, 1)};
\addplot[thick, dashed, black] coordinates {(1500,0) (1500, 2)};
\addplot[thick, dashed, black] coordinates {(3000,0) (3000, 2)};


\nextgroupplot[ylabel={\footnotesize Power \\ (Watts)}, % Power
ytick={0,50,100,150,200,250},
yticklabels={,50,100,150,200,250},
yticklabel style={font=\footnotesize},
ymin=50,
ymax=200,
xlabel={\footnotesize $time$ [frame]},
xlabel near ticks,
xtick={0,500,1000,1500,2000,2500,3000,3500,4000,4500},
xticklabels={0,,,1500,,,3000,,,4500},
xticklabel style={font=\footnotesize},
]
\addplot[thick, solid, color=s1] table[x index=0,y index=2,col sep=tab] {img/CoPPer/x264-phases-clover-copper.txt};
\addplot[thick, dashed, black] coordinates {(1500,0) (1500, 250)};
\addplot[thick, dashed, black] coordinates {(3000,0) (3000, 250)};

\end{groupplot}
\end{centering}

\end{tikzpicture}

  \caption{Processing x264 input with distinct phases using CoPPer.}
  \label{fig:copper-phases-x264}
\end{figure}

In \figref{copper-phases-x264}, we run the same experiment using \bench{x264} with CoPPer as we did with POET, without using a gain limit.
CoPPer is clearly able to meet the performance target, despite the noisy application and changes in phases.
CoPPer's overhead is $O(1)$ -- the controller runs in constant time as there is no need to search a configuration space.
The overhead of actuating system changes is also low -- power caps are applied at a socket level, as opposed to DVFS which requires setting each core (or set of related cores) separately.
Furthermore, CoPPer only requires one actuation in each window period, whereas POET usually requires two, as it must split a window period between two configurations.

\section{Energy Efficiency with Machine Learning}

A different goal in balancing performance and energy consumption is to run as energy-efficiently as possible.
More formally, an application (or perhaps a data center administrator) may wish maximize the amount of work completed per unit of energy consumption.
If energy efficiency is maximized, the system achieves an optimal throughput-to-cost ratio, where cost is measured in energy consumption, which has a known fiscal value.

In high-performance computing (HPC) environments, maximizing energy efficiency can allow more research to be completed for the same runtime costs.
Naturally there are fixed costs in operating large-scale computing clusters, with some components always, or almost always, running and thus consuming power.
It can therefore be beneficial to keep the rest of the data center busy as well.
Such environments are also often expected to consume a certain level of power that they contract for, and if that power is not used, it is wasted.
By maximizing energy efficiency in computation, more work can be performed simultaneously, particularly if all the compute nodes would not otherwise be used concurrently.
The same holds if the cluster is intentionally over-allocated, \ie more nodes exist than can be run at maximum power at any given moment.

Work related to this proposal (\secref{related}) indicates that statistical and machine learning approaches can be both useful and practical for managing energy consumption of HPC applications at runtime.
Significant related work is limited to building models for predicting power or energy consumption based on metrics like those captured from hardware performance counters.
The more practical of these works use their models to drive resource scheduling algorithms to reduce energy consumption, usually by tuning DVFS and/or thread concurrency.
% A very real benefit of statistical approaches is that we do not need to model complex computing systems, making well-designed techniques easier to reason about and more portable.
% Instead, a scheduler can infer behavior from sample and runtime data without having a priori knowledge of an entire tradeoff space or detailed knowledge of the system architecture.
% Of course we can still incorporate such knowledge to improve predictions, though.
To our knowledge, nobody has yet evaluated standard machine learning techniques for maximizing energy efficiency at runtime.
% \TODO{Introduce the concept of a micro-phase, where any change in system configuration caused by a change in application behavior would improve EE?}


\subsection{Proposal}

% \TODO{Why is this novel?}
Recalling the SEEC design, there are three tasks that a self-aware system must perform -- \emph{observe}, \emph{decide}, and \emph{act}.
To \emph{observe}, we use the Performance Counter Monitor (PCM) tool, originally created by Intel, now open source \cite{PCMGit}.
PCM exposes low-level hardware counter metrics relating to processor and memory hierarchy behavior, such as instructions retired, L2/L3 cache misses, and energy consumption.
While performance (\ie application progress) is most accurately measured by instrumenting an application with a tool like Heartbeats, this is not always feasible, nor does it work well for applications that do not contain high-level loops, or HPC applications that are difficult and time-consuming to modify.
We find that the rate of instructions retired is sufficient for estimating application progress, particularly when the performance value alone is not important to the decision logic (described shortly).
The measure of \emph{energy efficiency}, which we wish to maximize, is then $IPS / Watts$, where $IPS$ is instructions retired per second.
This reduces to instructions retired per Joule of energy consumed.\footnote{Some related work has targeted ``useful'' IPS, by detecting instructions that do not progress the application, like spinlocks or parallelization/synchronization instructions. If we find during the course of this project that this may be necessary, we can look into it further.}

We now propose to implement the \emph{decide} component by treating the decision as a classification problem.
Specifically, we wish to decide on the most energy-efficient DVFS setting to run in.
Many machine learning approaches are well-suited to this task, including but not limited to: support vector machine (SVM), k-nearest neighbors (KNN), random forest (RF), stochastic gradient descent (SGD), and gradient boosting (GB).
There are many off-the-shelf implementations of these algorithms for us to use, avoiding the need to implement them ourselves.
\figref{classifier-runtime} demonstrates the proposed approach, which again reflects the SEEC model.

\begin{figure}[t]
  \begin{centering}
    \tikzset{%
  app/.style    = {draw, thin, rectangle, minimum height = 2em,
    minimum width = 2em, fill=black!25},
  block/.style    = {draw, thick, rectangle, minimum height = 2.5em,
    minimum width = 2.5em},
  blockres/.style    = {draw, thick, rectangle, minimum height = 2.5em,
    minimum width = 2.5em, fill=green!25},
  biblock/.style  = {draw, thick, rectangle, minimum height = 5.5em,
    minimum width = 6em, fill=red!25},
  sum/.style      = {draw, circle, node distance = 2cm}, % Adder
  input/.style    = {coordinate}, % Input
  output/.style   = {coordinate} % Output
}

\begin{tikzpicture}[scale=1.0,transform shape, auto, thick, node distance=1.5cm, >=triangle 45]

\draw
  % Drawing the top blocks
  % node [input, name=goalaccuracy] {} 
  % node [left of=goalaccuracy, node distance=0.35mm]{}
  % node [sum, right of=goalaccuracy] (sumaccuracy) {} % negative feedback
  node [block, align=center] (featureselection) 
    {Feature\\Selection}
  node [block, right of=featureselection, align=center, node distance=3.5cm] (classifier) 
    {Classifier}
  node [blockres, above of=classifier, align=center, node distance=1.7cm] (trainingdata) 
    {Training\\Data}
;
  % Connectng lines
% \draw[->](goalaccuracy) -- node[align=center] {Timing\\Goal}(sumaccuracy);
% \draw[->](sumaccuracy) -- node[align=center] {Timing\\Error}(featureselection);
\draw[->](featureselection) -- node[align=center] {Processed\\Data}(classifier);
\draw[->](trainingdata) -- (classifier);

% Draw software system
\draw
  node [biblock, right of=classifier, node distance=4.2cm, align=center] (system)
    {\\System\\\\\\}
;
\draw
  node [app, right of=classifier, node distance=4.2cm, align=center, yshift=-0.5cm] (software)
    {Application}
;

% lines from translators to software
\draw[->](classifier.east) -- node [name=ka,align=center]{System\\Settings} (classifier.east -| system.west);

% Connectng lines
\coordinate (feedbackup) at ([yshift=-0.5cm]featureselection.south);
\draw (system.west |- feedbackup) -| node [near end,align=center] {PCM\\Sample} (feedbackup);
\draw[->](feedbackup) -- node[pos=0.99] {} (featureselection);

\end{tikzpicture}
    \caption{Overview of the machine learning classifier runtime.}
    \label{fig:classifier-runtime}
  \end{centering}
\end{figure}

We begin by capturing samples of PCM data from various benchmark applications by running them in different DVFS settings on a quad-socket, 160-logical-core server-class system.
For now, we limit ourselves to the 80 physical cores.
From these samples, we identify the most energy-efficient DVFS settings, which we use to train a machine learning \textbf{classifier}.
Before training, however, we must first normalize the PCM data and perform \textbf{feature selection} to identify fields (hardware counters) from the PCM data that correlate well with energy efficiency.
Initial indications are that \emph{principal component analysis} (PCA) will work well.
Once feature selection is performed, we train the classifier.

The classifier can then be run in the background of any application (or in the future, any set of applications).
It uses \emph{processed data} from PCM (post-normalization and feature selection) at preset time intervals and produces a prediction of the best \emph{DVFS frequency} to run in.
In the ideal case, the classifier will produce the optimal, \ie most energy-efficient, DVFS setting for the current application behavior, and change its prediction accordingly as the application behavior changes, \eg due to transitioning between phases/stages.
The \emph{act} component, as in prior controller work, is system-specific.

One example of applications we will test the classifiers with is High Performance Meraculous (HipMer), a distributed a scalable version of a genome assembler.
HipMer execution proceeds through a number of application phases, some of which have different optimal DVFS settings when maximizing energy efficiency.

\figref{classifier-phases-x264} shows a proof-of-concept of a SVM classifier providing predictions at runtime, using the same \bench{x264} application with phased input as demonstrated previously.
In this experiment we cannot track frames separately since the software is not instrumented with Heartbeats.
Instead, we present the DVFS frequency and energy efficiency achieved with each classifier iteration, which are at approximately one second intervals.
DVFS frequency is normalized to the system's nominal frequency, \ie the maximum, ignoring TurboBoost.
The horizontal bar indicates the optimal average frequency, computed offline, and energy efficiency is normalized to the average energy efficiency of this optimal frequency.

\begin{figure}[t]
  \begin{tikzpicture}
\begin{centering}

\definecolor{s1}{RGB}{228, 26, 28}
\definecolor{s2}{RGB}{55, 126, 184}
\definecolor{s3}{RGB}{77, 175, 74}
\definecolor{s4}{RGB}{152, 78, 163}
\definecolor{s5}{RGB}{255, 127, 0}

\begin{groupplot}[
    group style={
        group name=plots,
        group size=1 by 2,
        xlabels at=edge bottom,
        xticklabels at=edge bottom,
        vertical sep=5pt
    },
height=3.5cm,
width=0.95\columnwidth,
xmajorgrids,
ymajorgrids,
grid style={dashed},
xmin=0,
xmax=60,
yticklabel pos=left,
enlargelimits=false,
tick align = outside,
tick style={white},
xticklabel shift={-5pt},
yticklabel shift={-5pt},
ylabel shift={-2pt},
ylabel style={align=center},
unbounded coords=jump,
]

\nextgroupplot[ylabel={\footnotesize DVFS Frequency \\ (Normalized)}, % Performance
xtick={0,5,10,15,20,25,30,35,40,45,50,55,60},
ytick={0.0,0.2,0.4,0.6,0.8,1.0},
yticklabels={0.0,0.2,0.4,0.6,0.8,1.0},
yticklabel style={font=\footnotesize},
ymin=0.5,
ymax=1.1,
% legend entries={{\footnotesize $\mathsf{Server}$}},
% legend style={draw=none,at={(0.5,1.4)},anchor=north,legend columns=4,line width=5pt},
]
\addplot[thick, solid, color=s2] table[x index=0,y index=1,col sep=tab] {img/boggle/x264-phases-boggle-classifier.txt};
\addplot[thick, solid, black] coordinates {(0, 1) (60, 1)};
% \addplot[thick, dashed, black] coordinates {(1500,0) (1500, 2)};
% \addplot[thick, dashed, black] coordinates {(3000,0) (3000, 2)};


\nextgroupplot[ylabel={\footnotesize Energy Efficiency \\ (Normalized)}, % Power
ytick={0.0,0.2,0.4,0.6,0.8,1.0,1.2,1.4},
yticklabels={0.0,0.2,0.4,0.6,0.8,1.0,1.2,1.4},
yticklabel style={font=\footnotesize},
ymin=0,
ymax=1.4,
xlabel={\footnotesize $time$ [seconds]},
xlabel near ticks,
xtick={0,5,10,15,20,25,30,35,40,45,50,55,60},
xticklabels={0,,10,,20,,30,,40,,50,,60},
xticklabel style={font=\footnotesize},
]
\addplot[thick, solid, color=s1] table[x index=0,y index=2,col sep=tab] {img/boggle/x264-phases-boggle-classifier.txt};
\addplot[thick, solid, black] coordinates {(0, 1) (60, 1)}; % race
% \addplot[thick, solid, black] coordinates {(0, 0.773) (60, 0.773)}; % worst-case
% \addplot[thick, dashed, black] coordinates {(1500,0) (1500, 250)};
% \addplot[thick, dashed, black] coordinates {(3000,0) (3000, 250)};

\end{groupplot}
\end{centering}

\end{tikzpicture}

  \caption{Processing x264 input with distinct phases using a Support Vector Machine classifier.}
  \label{fig:classifier-phases-x264}
\end{figure}

There is clearly still room for improvement, but offline analysis indicates that the the proposed machine learning approach is promising, and we have demonstrated that such an approach is feasible on a real system.
It should also be noted that \bench{x264} does not scale particularly well on this system for this input, so we expect to see better results for more HPC-like applications.
The remaining work in the thesis includes:
\begin{itemize}
\item Examining in further detail which PCM fields are most useful in predicting energy-efficient configurations.
% \item Exploring feature selection techniques beyond PCA.
\item Evaluating the aforementioned classification approaches, and others not yet discussed, on one or more server-class systems at Lawrence Berkeley National Laboratory.
\item Evaluating multiple-application scenarios, in which co-located applications run concurrently.
\item Improving energy efficiency further through additional actuators that we find to be beneficial, specifically at least one of: socket allocation, Hyperthreading, or using power capping instead of DVFS.
\end{itemize}
Socket and/or core allocation, \eg Hyperthreads, should provide additional improvements in energy efficiency for some applications, particularly those that do not scale well (like \bench{x264}).
% To do this properly, we may need additional support from applications, OpenMP and MPI implementations to provide truly dynamic thread management at runtime.
As specified as a motivator for the CoPPer project, adjusting power caps instead of DVFS appears to be the future approach for software to explicitly manage power (and thus frequency) allocations.
Power capping could also be treated as a classification problem by limiting the power caps to a discrete set of values (\eg in RAPL, power caps are indeed limited to fixed increments), though the number of possible settings would likely be large for server-class systems.
It could also be treated as an estimation problem, in which the decision logic produces a real value (or vector of values) to apply to one or more system components.
These tasks will almost certainly need to be iterated on multiple times, but we are confident there is publishable research insight.

% \TODO{Some outstanding problems to address.  For example, PCM data being corrupted by also capturing the classifier's compute cycles, the need for "useful" IPS as opposed to raw IPS for measuring performance.}


\subsection{Related Work}
\label{sec:related}

% \TODO{Must be able to speak in more detail about any work cited here.}

There is plenty of prior work in power-aware scheduling for HPC.
Significant research has come out of LLNL, especially from from B.~Rountree, M.~Schulz, B.~de Supinski, and their collaborators like D.~Lowenthal from the University of Arizona and his current and former students.
For example, Rountree \etal propose Adagio, which uses DVFS for saving energy in HPC applications with less than 1\% increase in runtime \cite{RountreeAdagio}.
Adagio depends on accurate time predictions of both computation and MPI communication and works by predicting critical path code, only allowing slowdown for work that is not on the critical path.
Follow-on work by Marathe \etal combines Adagio with Jitter \cite{Jitter} to distribute power to nodes and cores to improve HPC application performance, and also depends on accurate prediction of the critical path \cite{Marathe2015}.
In multiple publications, Patki \etal propose hardware over-provisioning while operating under power constraints to increase total system throughput with approaches like RMAP \cite{PatkiRMAP}.
As stated earlier, over-provisioned environments are good places for maximizing energy efficiency, and more recent work has begun to put over-provisioning into practice.

Other works related to performance and energy consumption make use of hardware performance counters, as we propose to do.
Wu \etal use performance/power modeling to provide suggestions for application changes to improve energy efficiency \cite{WuHPCComputer}.
Importantly, they use Principle Component Analysis (PCA) and Spearman correlation to identify correlated performance counters related to energy efficiency, and multivariate regression to generate performance and power models.
% These models then drive a what-if prediction mechanism, \eg if TLB misses can be reduced by some amount, what will be the effect on performance and power?
Tsafack Chetsa \etal use hardware performance counters to both predict an application's peak energy consumption and to reduce energy consumption of applications in HPC environments \cite{Chetsa}.
Rather than focusing on phases of individual applications, they model the behavior of each node in a HPC system, treating it as a black box.
This is not unlike our proposal in which we measure energy efficiency at a system level and will eventually evaluate multi-application scenarios.
As a prerequisite for this thesis proposal, their work establishes, by citing existing publications, that performance counters can indeed be used to accurately model power/energy consumption of a whole system or specific system components like CPU and memory.
Libutti \etal use a multi-objective evaluation approach to select and map resources to co-schedule applications and increase energy-delay-product \cite{Libutti2014}; performance counters are used to correlate memory intensiveness and power consumption, and are chosen statically.

Sasaki \etal use a statistical learning and analysis approach to dynamically reduce energy consumption, up to a user-specified loss-in-performance ratio, for application phases determined at compile time \cite{Sasaki}.
Shye \etal analyze the relationship between architectural performance counters and user satisfaction for applications, and use artificial neural networks to predict DVFS settings that will satisfy users while reducing power consumption \cite{ShyeIDVFS}.
This approach is interesting in that they use more advanced machine learning mechanisms, but its evaluation metrics are highly user and application-dependent.
In a short paper, Alvarado \etal use supervised learning techniques to create heuristics for managing thread processor affinity to improve energy consumption of parallel HPC applications based on a limited set of performance counters \cite{Alvarado}.
The paper briefly discusses related work in statistical and machine learning techniques for scheduling thread affinity, but much of them are not power/energy-aware.

The most similar work we are aware of is by Curtis-Maury \etal which uses a logistic regression model that is energy-aware, predicting performance and power to manage both Dynamic Concurrency Throttling (DCT) and DVFS to reduce energy consumption without performance loss (and sometimes with performance gain) \cite{Curtis-Maury2008}.
Later work by Li \etal supports hybrid applications that combine OpenMP with MPI in distributed environments, requiring additional modeling and for memory, communication, and parallelization \cite{LiIPDPS2010}.
Both works make use of hardware events, in combination with their own models using coefficients computed with multivariate regression.
These works develop new models specifically designed to schedule applications for improved energy consumption with little or no performance loss, whereas our proposal is to use standard machine learning techniques to optimally balance performance and energy consumption by maximizing energy efficiency, without the need for complex profiling or prediction models.
% \TODO{These papers' related works sections are good.}

Delimitrou and Kozyrakis propose Paragon, an interference and heterogeneity-aware scheduler for datacenters \cite{Paragon}, which uses an approach similar to the Netflix algorithm \cite{NetflixPrize}.
Paragon uses data from previously scheduled applications to perform offline training of a classifier.
This classifier can quickly measure the behavior of a new application at runtime and predict what server configuration it will run best on (heterogeneity in the datacenter), its impact on other co-located applications, and its own ability to tolerate interference.
% Furthermore, Paragon scales to tens of thousands of servers and dozens of configurations.
This work provides evidence that a classification approach backed by general machine learning and statistical techniques can make accurate resource scheduling predictions, as we propose to do for maximizing energy efficiency.
We are hopeful that this general approach can produce results as good as the aforementioned works by Curtis-Maury \etal and Li \etal without the need for their specialized architectural-based statistical models and library instrumentation.


\subsection{Exploratory and Post-Thesis Work}

There is plenty of follow-up work and possible extensions to this proposal that we would also like to explore.
The most interesting research questions arise when considering the balance of performance and energy consumption across multiple systems (nodes), which we typically find in both HPC and cloud computing environments.
Additional complexity arises in bridging the gap between local optimization and global optimization.
Such problems may include, but are not limited to: the possible need for a centralized master controller (thus limiting scalability), increased time and energy overhead of communication between nodes (linear or even exponential in simple approaches), and avoiding decisions that seem beneficial locally but are counter-productive globally.
Further issues arise when considering heterogeneous environments, unbalanced workloads, node failure scenarios, application check-pointing, and shared resources like the network and non-local storage.


% % Format a LaTeX bibliography
\bibliography{seec}
% \makebibliography

\end{document}
