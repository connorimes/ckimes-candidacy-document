\section{Performance and Energy Management using Control Theory}

We first address the problem of meeting application performance goals while minimizing energy consumption.
Many applications are subject to soft performance constraints -- for example, streaming applications often consume data from other sources like sensors, or must meet QoS requirements when serving client requests.
In this project, we restrict the application domain to streaming-like applications with high-level loops, to which we apply the concept of Heartbeats to measure and record application performance and system power behavior \cite{icac2010heartbeats}.
Practically, this involves adding a small amount of instrumentation into the applications to capture these metrics at runtime.
This is the \emph{observe} component of a SEEC-like system.
To support this observation requirement, we complement Heartbeats with the EnergyMon interface for capturing power/energy data in a portable manner \cite{energymon}.\footnote{EnergyMon with Java and Rust bindings and abstractions available at https://github.com/energymon/}

Unlike heuristics, control theory provides a formal approach for meeting goals in dynamic systems.
Control-theoretic approaches can adapt despite non-trivial errors in their design parameters and guarantee convergence under certain bounds.
% There has been significant prior work \TODO{describe} in using designing software controllers, but most lack portability -- they are designed for specific systems and applications.
The \emph{decide} component of a SEEC-like system is described by the following controllers, POET and CoPPer.
Finally, the \emph{act} component is platform-specific, and typically involves actuating system settings like core allocation and processor DVFS frequencies or power limits.


\subsection{POET}

Using \cite{Imes2014} as motivation for needing a portable approach to minimizing energy under performance constraints, we designed the Performance with Optimal Energy Toolkit (POET).
POET meets soft real-time constraints and minimizes energy consumption by actuating DVFS frequencies and core allocation \cite{POET}.
While POET was originally evaluated on embedded systems, we later evaluated it on a dual socket, 32-logical-core server-class system \cite{POETMCSoC}.
Its design, shown in \figref{poet-runtime}, reflects the SEEC model.

\begin{figure}[t]
  \begin{centering}
    \tikzset{%
  app/.style    = {draw, thin, rectangle, minimum height = 2em,
    minimum width = 2em, fill=black!25},
  block/.style    = {draw, thick, rectangle, minimum height = 2.5em,
    minimum width = 2.5em},
  blockres/.style    = {draw, thick, rectangle, minimum height = 2.5em,
    minimum width = 2.5em, fill=green!25},
  biblock/.style  = {draw, thick, rectangle, minimum height = 5.5em,
    minimum width = 6em, fill=red!25},
  sum/.style      = {draw, circle, node distance = 2cm}, % Adder
  input/.style    = {coordinate}, % Input
  output/.style   = {coordinate} % Output
}

\begin{tikzpicture}[scale=1.0,transform shape, auto, thick, node distance=1.5cm, >=triangle 45]

\draw
  % Drawing the top blocks
  node [input, name=goalaccuracy] {} 
  node [left of=goalaccuracy, node distance=0.35mm]{}
  node [sum, right of=goalaccuracy] (sumaccuracy) {} % negative feedback
  node [block, right of=sumaccuracy, align=center, node distance=2.6cm] (controlaccuracy) 
    {Controller}
  node [block, right of=controlaccuracy, align=center, node distance=3.5cm] (translateaccuracy) 
    {Optimizer}
  node [blockres, above of=translateaccuracy, align=center, node distance=1.7cm] (resourcefile) 
    {Resource\\Specification}
;
  % Connectng lines
\draw[->](goalaccuracy) -- node[align=center] {Performance\\Goal}(sumaccuracy);
\draw[->](sumaccuracy) -- node[align=center] {Performance\\Error}(controlaccuracy);
\draw[->](controlaccuracy) -- node[align=center] {Generic\\Control\\Signal}(translateaccuracy);
\draw[->](resourcefile) -- (translateaccuracy);

% Draw software system
\draw
  node [biblock, right of=translateaccuracy, node distance=4.2cm, align=center] (system)
    {\\System\\\\\\}
;
\draw
  node [app, right of=translateaccuracy, node distance=4.2cm, align=center, yshift=-0.5cm] (software)
    {Application}
;

% lines from translators to software
\draw[->](translateaccuracy.east) -- node [name=ka,align=center]{Resource\\Schedule} (translateaccuracy.east -| system.west);

% Connectng lines
\coordinate (feedbackup) at ([yshift=-0.6cm]sumaccuracy.south);
\draw (software.west |- feedbackup) -| node [near end,align=center] {Performance\\Feedback} (feedbackup);
\draw[->](feedbackup) -- node[pos=0.99] {$-$} (sumaccuracy);

\end{tikzpicture}
    \caption{Overview of the POET runtime.}
    \label{fig:poet-runtime}
  \end{centering}
\end{figure}

In POET, the user provides a \emph{performance goal}, $P_{ref}$, to the controller; the user or a systems developer provides it with a \textbf{resource specification}, which is a normalized table of performance and power behavior derived from a characterization of a representative application.
At fixed work intervals called window periods, the controller computes a new \emph{resource schedule} that is applied over the following window period.

The schedule is computed by first calculating the \emph{performance error}, $e(t)$, between the performance goal, and the observed application performance (\emph{performance feedback}), $p_m(t)$:
\begin{eqnarray}
  e(t) = P_{ref} - p_m(t)
  \label{eqn:error}
\end{eqnarray}
The \textbf{controller} uses this error to compute a new speedup value (\emph{generic control signal}), $s(t)$, so that the speedup cancels the error:
\begin{equation}
  s(t) = s(t-1) + (1-p) \cdot \frac{e(t)}{b(t)} 
  \label{eqn:poet-control}
\end{equation}
Here $p$ is a configurable \emph{pole} value such that $0 \le p < 1$, which affects how responsive the controller is to system noise or changes in application behavior.
A small value of $p$ causes the controller to respond quickly, but is also more susceptible to noise.
A larger $p$ slows the response, but ensures robustness and is more useful in noisy systems.

Because the POET's resource specification is normalized to $c=0$ (described shortly), the controller must estimate the application's base speed, $b(t)$, to bridge the gap between speedup and actual performance.
Every application has its own base speed, whose value may change at runtime as the application progresses through different \emph{phases}.
POET updates its base speed estimate at every control iteration using a Kalman filter~\cite{welch2006kalman}.

Once $s(t)$ is computed, POET's \textbf{optimizer} searches the resource specification to find the pair of configurations that provides the optimal energy consumption while respecting the performance (speedup) constraint.
There are $C$ configurations in the resource specification, and by convention we number the configurations from $0$ to $C-1$.
Configuration $c = 0$ is the lowest-performance/lowest-power configuration where the least amount of resources are allocated.
Configuration $C-1$ is the highest-performance/highest-power configuration where the maximum amount of resources are allocated.
Every configuration $c$ has normalized performance (speedup) $s_c$ and normalized power consumption (powerup) $p_c$ values.

In computing the resource schedule, the optimizer assigns each configuration $c$ an execution time $\tau_c$, ensuring that the $W$ iterations complete while the total energy is minimized, where $W$ is the window period.
This optimization problem is formulated as a linear program, solved in $O(C^2)$ time:
\begin{eqnarray}
\minimize && \sum_{c=0}^{C-1} \tau_c \cdot p_c \label{eqn:poet-power} \\
\st %&& \nonumber\\
&& \sum_{c=0}^{C-1} \tau_c \cdot s_c \cdot b(t) = W \label{eqn:poet-work} \\
&& \sum_{c=0}^{C-1} \tau_c =  \tau \label{eqn:poet-deadline} \\
&& 0 \le \tau_c \le \tau, \qquad \forall c \in \{0,\ldots,C-1\} \label{eqn:poet-time}
\end{eqnarray}
The resulting schedule consists of, at most, two configurations with $\tau_c > 0$.

This control and optimization approach provides formal guarantees about steady-state convergence and robustness while remaining independent of any particular application or system.
We demonstrated POET's ability to meet performance goals and adapt to changes in application phases at runtime by executing the \bench{x264} video encoding application using an input that is a combination of three videos of varying encoding difficulty.

\figref{poet-phases-default} is a time series of job performance and system power consumption when running \bench{x264} without any control in the highest resource configuration ($C-1$).
Performance is normalized to the maximum recorded value.
Frames that are easier to encode (higher performance) take less time, indicating that it would require fewer system resources to meet a performance goal than a frame with lower performance.
Input phase changes occur at frames 1,500 and 3,000, where there are observable changes in performance.
The first and third phases have similar performance behavior, and thus are about the same level of encoding difficulty; the second phase is easiest to encode as it has higher average performance.

\begin{figure}[t]
  \begin{tikzpicture}
\begin{centering}

\definecolor{s1}{RGB}{228, 26, 28}
\definecolor{s2}{RGB}{55, 126, 184}
\definecolor{s3}{RGB}{77, 175, 74}
\definecolor{s4}{RGB}{152, 78, 163}
\definecolor{s5}{RGB}{255, 127, 0}

\begin{groupplot}[
    group style={
        group name=plots,
        group size=1 by 2,
        xlabels at=edge bottom,
        xticklabels at=edge bottom,
        vertical sep=5pt
    },
height=3.5cm,
width=0.95\columnwidth,
xmajorgrids,
ymajorgrids,
grid style={dashed},
xmin=0,
xmax=4500,
yticklabel pos=left,
enlargelimits=false,
tick align = outside,
tick style={white},
xticklabel shift={-5pt},
yticklabel shift={-5pt},
ylabel shift={-2pt},
ylabel style={align=center},
unbounded coords=jump,
]

\nextgroupplot[ylabel={\footnotesize Performance \\ (Normalized)}, % Performance
xtick={0,500,1000,1500,2000,2500,3000,3500,4000,4500},
ytick={0,0.2,0.4,0.6,0.8,1.0,1.2},
yticklabels={,0.2,0.4,0.6,0.8,1.0,1.2},
yticklabel style={font=\footnotesize},
ymin=0,
ymax=1.0,
% legend entries={{\footnotesize $\mathsf{Server}$}},
% legend style={draw=none,at={(0.5,1.35)},anchor=north,legend columns=4,line width=5pt},
]
\addplot[thick, solid, color=s2] table[x index=0,y index=1,col sep=tab] {img/clover/x264-phases-default-clover.txt};
\addplot[thick, dashed, black] coordinates {(1500,0) (1500, 2)};
\addplot[thick, dashed, black] coordinates {(3000,0) (3000, 2)};


\nextgroupplot[ylabel={\footnotesize Power \\ (Watts)}, % Power
ytick={0,50,100,150,200,250},
yticklabels={,50,100,150,200,250},
yticklabel style={font=\footnotesize},
ymin=50,
ymax=250,
xlabel={\footnotesize $time$ [frame]},
xlabel near ticks,
xtick={0,500,1000,1500,2000,2500,3000,3500,4000,4500},
xticklabels={0,,,1500,,,3000,,,4500},
xticklabel style={font=\footnotesize},
]
\addplot[thick, solid, color=s1] table[x index=0,y index=2,col sep=tab] {img/clover/x264-phases-default-clover.txt};
\addplot[thick, dashed, black] coordinates {(1500,0) (1500, 250)};
\addplot[thick, dashed, black] coordinates {(3000,0) (3000, 250)};

\end{groupplot}
\end{centering}

\end{tikzpicture}
  \caption{Processing x264 input with distinct phases in an uncontrolled setting.}
  \label{fig:poet-phases-default}
\end{figure}

\begin{figure}[t]
  \begin{tikzpicture}
\begin{centering}

\definecolor{s1}{RGB}{228, 26, 28}
\definecolor{s2}{RGB}{55, 126, 184}
\definecolor{s3}{RGB}{77, 175, 74}
\definecolor{s4}{RGB}{152, 78, 163}
\definecolor{s5}{RGB}{255, 127, 0}

\begin{groupplot}[
    group style={
        group name=plots,
        group size=1 by 2,
        xlabels at=edge bottom,
        xticklabels at=edge bottom,
        vertical sep=5pt
    },
height=3.5cm,
width=0.95\columnwidth,
xmajorgrids,
ymajorgrids,
grid style={dashed},
xmin=0,
xmax=4500,
yticklabel pos=left,
enlargelimits=false,
tick align = outside,
tick style={white},
xticklabel shift={-5pt},
yticklabel shift={-5pt},
ylabel shift={-2pt},
ylabel style={align=center},
unbounded coords=jump,
]

\nextgroupplot[ylabel={\footnotesize Latency \\ (Normalized)}, % Performance
xtick={0,500,1000,1500,2000,2500,3000,3500,4000,4500},
ytick={0.0,0.5,1.0,1.5,2.0},
yticklabels={,0.5,1.0,1.5,2.0},
yticklabel style={font=\footnotesize},
ymin=0,
ymax=2,
% legend entries={{\footnotesize $\mathsf{Server}$}},
% legend style={draw=none,at={(0.5,1.4)},anchor=north,legend columns=4,line width=5pt},
]
\addplot[thick, solid, color=s2] table[x index=0,y index=1,col sep=tab] {img/POET/x264-phases-clover-seec.txt};
\addplot[thick, solid, black] coordinates {(0, 1) (4500, 1)};
\addplot[thick, dashed, black] coordinates {(1500,0) (1500, 2)};
\addplot[thick, dashed, black] coordinates {(3000,0) (3000, 2)};


\nextgroupplot[ylabel={\footnotesize Power \\ (Watts)}, % Power
ytick={0,50,100,150,200,250},
yticklabels={,50,100,150,200,250},
yticklabel style={font=\footnotesize},
ymin=50,
ymax=250,
xlabel={\footnotesize $time$ [frame]},
xlabel near ticks,
xtick={0,500,1000,1500,2000,2500,3000,3500,4000,4500},
xticklabels={0,,,1500,,,3000,,,4500},
xticklabel style={font=\footnotesize},
]
\addplot[thick, solid, color=s1] table[x index=0,y index=2,col sep=tab] {img/POET/x264-phases-clover-seec.txt};
\addplot[thick, dashed, black] coordinates {(1500,0) (1500, 250)};
\addplot[thick, dashed, black] coordinates {(3000,0) (3000, 250)};

\end{groupplot}
\end{centering}

\end{tikzpicture}

  \caption{Processing x264 input with distinct phases using POET.}
  \label{fig:poet-phases-x264}
\end{figure}

In \figref{poet-phases-x264}, we run with POET enabled, using a performance target that is about half of the system's maximum capacity.
POET takes initial performance measurements during the first window period of 100 frames, then begins changing system resources to meet the performance goal.
After the end of the third window period, around frame 300, the soft performance goal is respected for the remainder of the runtime.
Performance is still somewhat noisy due to the unpredictable behavior of \bench{x264}, as can be observed in the uncontrolled execution, with slight dips and peaks appearing at the beginning of phase changes before POET has time to adapt.

POET has also been used as the foundation for other controllers and projects.
We expanded on POET to create Bard, which adds the ability to instead meet soft power constraints and maximize performance \cite{Bard}.\footnote{Both POET and Bard are available at http://poet.cs.uchicago.edu/}
Farrell and Hoffmann build on POET by additionally changing application accuracy to provide hard real-time guarantees \cite{meantime}.
In work still under review, Mishra \etal combine POET with machine learning to reduce the amount of offline work needed to generate resource specifications, and update both the specification and pole value online as new application behavior is learned.
POET was also the starting point in an ongoing project called Proteus, in which we are developing a programming language called FAST that allows programmers to specify more general constraints and optimizations called \emph{intents} -- both system and application knobs can be adjusted to satisfy the intents.


\subsection{CoPPer}

As hardware manufacturers move away from software-specified DVFS settings, we instead propose using software-defined, hardware-enforced power capping as the actuation mechanism for controllers to balance performance and power/energy consumption.
We therefore proposed CoPPer, \textbf{Co}ntrol \textbf{P}erformance with \textbf{P}ow\textbf{er}.\footnote{CoPPer and related tools will be available at https://github.com/powercap/}
In CoPPer, we address the same problem of minimizing energy consumption under soft performance constraints, with the hardware power capping implementation performing DVFS optimization ``under the hood'' for us, avoiding the need for the software controller to compute an optimal schedule.
For this project, we do not manage core allocation.
We evaluated CoPPer on the same server-class system as POET, using Intel Running Average Power Limit (RAPL) as the power capping implementation \cite{RAPL}.
Specifically, we use the short term Package-level power constraints on each of the two system processor sockets.

\begin{figure}[t]
  \begin{centering}
    \tikzset{%
  app/.style    = {draw, thin, rectangle, minimum height = 2em,
    minimum width = 2em, fill=black!25},
  block/.style    = {draw, thick, rectangle, minimum height = 2.5em,
    minimum width = 2.5em},
  blockres/.style    = {draw, thick, rectangle, minimum height = 2.5em,
    minimum width = 2.5em, fill=green!25},
  biblock/.style  = {draw, thick, rectangle, minimum height = 5.5em,
    minimum width = 6em, fill=red!25},
  sum/.style      = {draw, circle, node distance = 2cm}, % Adder
  input/.style    = {coordinate}, % Input
  output/.style   = {coordinate} % Output
}

\begin{tikzpicture}[scale=1.0,transform shape, auto, thick, node distance=1.5cm, >=triangle 45]

\draw
  % Drawing the top blocks
  node [input, name=goalaccuracy] {} 
  node [left of=goalaccuracy, node distance=0.35mm]{}
  node [sum, right of=goalaccuracy] (sumaccuracy) {} % negative feedback
  node [block, right of=sumaccuracy, align=center, node distance=2.9cm] (controlaccuracy) 
    {~Controller~}
  % node [block, right of=controlaccuracy, align=center, node distance=3.0cm] (translateaccuracy) 
  %   {Mapper}
  node [blockres, above of=controlaccuracy, align=center, node distance=1.8cm] (resourcefile) 
    {Min / Max\\Power}
;
  % Connectng lines
\draw[->](goalaccuracy) -- node[align=center] {Performance\\Goal}(sumaccuracy);
\draw[->](sumaccuracy) -- node[align=center] {Performance\\Error}(controlaccuracy);
% \draw[->](controlaccuracy) -- node[align=center] {Generic\\Control\\Signal}(translateaccuracy);
\draw[->](resourcefile) -- (controlaccuracy);

% Draw software system
\draw
  node [biblock, right of=controlaccuracy, node distance=4.2cm, align=center] (system)
    {\\System\\\\\\}
;
\draw
  node [app, right of=controlaccuracy, node distance=4.2cm, align=center, yshift=-0.5cm] (software)
    {Application}
;

% lines from translators to software
\draw[->](controlaccuracy.east) -- node [name=ka,align=center]{Power Cap} (controlaccuracy.east -| system.west);

% Connectng lines
\coordinate (feedbackup) at ([yshift=-0.6cm]sumaccuracy.south);
\draw (software.west |- feedbackup) -| node [near end,align=center] {\\Performance\\Feedback} (feedbackup);
\draw[->](feedbackup) -- node[pos=0.99] {$-$} (sumaccuracy);

\end{tikzpicture}
    \caption{Overview of the CoPPer runtime.}
    \label{fig:copper-runtime}
  \end{centering}
\end{figure}

\figref{copper-runtime} demonstrates CoPPer's design, which also reflects the SEEC model.
Note the absence of an \textbf{optimizer} and \textbf{resource specification}.
Unlike POET, CoPPer does not require a user or developer-specified model based on an application characterization -- only minimum and maximum power caps are needed.
While CoPPer uses a similar control-theoretic approach as POET, it introduces a \emph{gain limit} term into the speedup formulation.
In layman's terms, the gain limit ``pulls'' down the power cap to prevent over-allocating power when it is not beneficial.

As in POET, the CoPPer first calculates the error, $e(t)$, between desired and actual performance and estimates the base speed, $b(t)$ using a Kalman filter.
The speedup formulation used in CoPPer is then:
\begin{eqnarray}
  s(t) = max\left(1, gain(t) \cdot min\left(s(t-1) + \frac{e(t)}{b(t)}, \frac{U_{max}}{U_{min}}\right)\right)
  \label{eqn:copper-speedup-control}
\end{eqnarray}
where $U_{min}$ and $U_{max}$ are the system's minimum and maximum allowable power caps, respectively, and $0 < gain(t) <= 1$.
Using $U_{min}$ and $U_{max}$ to clamp the upper bound of $s(t)$ before applying the gain prevents slow controller response when the performance goal is not achievable (in control theory terminology, this is called an anti-windup mechanism).
Note that the speedup signal is now strictly internal to the \textbf{controller}, and is translated to a \emph{power cap} by scaling $s(t)$ by $U_{min}$.
To maintain convergence guarantees, $gain(t)$ is set to $1$ until the controller settles, regardless of the user configuration.

To formally describe the gain, we quote from the CoPPer paper (still under review):
\begin{quote}
Intuitively, if the performance error value [...] is low, then the system
has converged to the performance target and the speedup signal should
remain where it is.  However, if error values are high but the
\emph{difference} in error values between iterations is low, the controller
has settled, but the performance target is not achievable.  It
therefore may be beneficial to reduce the speedup, and thus the power.
Speedup is reduced by setting gain to:
\begin{eqnarray}
  gain(t) = 1 - \alpha_c \cdot e_{ns}(t) \cdot \Delta e_{ns}(t)
  \label{eqn:cost-pole}
\end{eqnarray}
where $\alpha_c$ ($0 \le \alpha_c < 1$) is the \emph{gain limit}, a
constant that controls how low the gain can go, and:
\begin{eqnarray}
  e_n(t) = \frac{|e(t)|}{P_{ref}}
  \label{eqn:en} \\
  \Delta e_n(t) = \left| e_n(t-1) - e_n(t) \right|
  \label{eqn:den} \\
  e_{ns}(t) = 1 - \frac{1}{e_{n}(t)+1}
  \label{eqn:ens} \\
  \Delta e_{ns}(t) = \frac{1}{\Delta e_{n}(t)+1}
  \label{eqn:dens}
\end{eqnarray}
Since $P_{ref}$ is the performance target, $e_n(t)$ is the absolute normalized performance error.
$\Delta e_n(t)$ in \eqnref{den} is the absolute change in $e_n(t)$ since the previous
iteration.  \eqnsref{ens}{dens} compute values $e_{ns}(t)$ and $\Delta
e_{ns}(t)$, which determine how much impact $e_n(t)$ and $\Delta
e_{n}(t)$ have on reducing the speedup.  Both $e_{ns}(t)$ and $\Delta
e_{ns}(t)$ lay in the unit circle.  As normalized performance error
$e_n(t)$ approaches 0, $e_{ns}(t)$ also approaches 0, which reduces
the impact of the gain limit in \eqnref{cost-pole}.  Conversely, if
the error is high, $e_{ns}(t)$ approaches 1 and the gain limit will
have a greater impact on the change speedup.  As the change in
normalized performance error $\Delta e_{n}(t)$ approaches 0, $\Delta
e_{ns}(t)$ approaches 1, thus increasing the gain limit's impact in
\eqnref{cost-pole}.  Conversely, if the change in error is high,
$\Delta e_{ns}(t)$ approaches 0 and the gain limit will have less
impact on the change in speedup.  Therefore, \eqnref{cost-pole}
reduces the speedup signal by a factor of at most $\alpha_c$, with the
greatest change in speedup occurring when the absolute performance
error $e_n(t)$ is high and the absolute change in error $e_n(t)$ is
low.  Setting $\alpha_c=0$ disables the gain limit entirely,
corresponding to $gain(t) =1 $ in \eqnref{copper-speedup-control}.
\end{quote}

\begin{figure}[t]
  \begin{tikzpicture}
\begin{centering}

\definecolor{s1}{RGB}{228, 26, 28}
\definecolor{s2}{RGB}{55, 126, 184}
\definecolor{s3}{RGB}{77, 175, 74}
\definecolor{s4}{RGB}{152, 78, 163}
\definecolor{s5}{RGB}{255, 127, 0}

\begin{groupplot}[
    group style={
        group name=plots,
        group size=1 by 2,
        xlabels at=edge bottom,
        xticklabels at=edge bottom,
        vertical sep=5pt
    },
height=3.5cm,
width=0.95\columnwidth,
xmajorgrids,
ymajorgrids,
grid style={dashed},
xmin=0,
xmax=4500,
yticklabel pos=left,
enlargelimits=false,
tick align = outside,
tick style={white},
xticklabel shift={-5pt},
yticklabel shift={-5pt},
ylabel shift={-2pt},
ylabel style={align=center},
unbounded coords=jump,
]

\nextgroupplot[ylabel={\footnotesize Performance \\ (Normalized)}, % Performance
xtick={0,500,1000,1500,2000,2500,3000,3500,4000,4500},
ytick={0.0,0.5,1.0,1.5,2.0},
yticklabels={,0.5,1.0,1.5,2.0},
yticklabel style={font=\footnotesize},
ymin=0,
ymax=2,
% legend entries={{$\mathsf{DVFS}$},{$\mathsf{\SYSTEM{}}$}},
% legend style={draw=none,at={(0.5,1.4)},anchor=north,legend columns=4,line width=5pt},
]
\addplot[thick, solid, color=s2] table[x index=0,y index=1,col sep=tab] {img/CoPPer/x264-phases-clover-copper.txt};
\addplot[thick, solid, black] coordinates {(0, 1) (4500, 1)};
\addplot[thick, dashed, black] coordinates {(1500,0) (1500, 2)};
\addplot[thick, dashed, black] coordinates {(3000,0) (3000, 2)};


\nextgroupplot[ylabel={\footnotesize Power \\ (Watts)}, % Power
ytick={0,50,100,150,200,250},
yticklabels={,50,100,150,200,250},
yticklabel style={font=\footnotesize},
ymin=50,
ymax=200,
xlabel={\footnotesize $time$ [frame]},
xlabel near ticks,
xtick={0,500,1000,1500,2000,2500,3000,3500,4000,4500},
xticklabels={0,,,1500,,,3000,,,4500},
xticklabel style={font=\footnotesize},
]
\addplot[thick, solid, color=s1] table[x index=0,y index=2,col sep=tab] {img/CoPPer/x264-phases-clover-copper.txt};
\addplot[thick, dashed, black] coordinates {(1500,0) (1500, 250)};
\addplot[thick, dashed, black] coordinates {(3000,0) (3000, 250)};

\end{groupplot}
\end{centering}

\end{tikzpicture}

  \caption{Processing x264 input with distinct phases using CoPPer.}
  \label{fig:copper-phases-x264}
\end{figure}

In \figref{copper-phases-x264}, we run the same experiment using \bench{x264} with CoPPer as we did with POET, without using a gain limit.
CoPPer is clearly able to meet the performance target, despite the noisy application and changes in phases.
CoPPer's overhead is $O(1)$ -- the controller runs in constant time as there is no need to search a configuration space.
The overhead of actuating system changes is also low -- power caps are applied at a socket level, as opposed to DVFS which requires setting each core (or set of related cores) separately.
Furthermore, CoPPer only requires one actuation in each window period, whereas POET usually requires two, as it must split a window period between two configurations.
